\chapter{Conclusions and future works}
As outlined in the introduction, this thesis sets out to develop a platform that handles key
operations of an Enterprise Resource Planning system while incorporating microservices and
serverless computing design patterns, ensuring accessibility across various platforms. With the
completion of the platform's initial version, we have established a technical foundation that
supports the ongoing development of diverse modules and functionalities. The platform is designed
for easy extensibility and scalability, making it adaptable to a range of needs and requirements.
Currently, the platform can be deployed on an AWS tenant using the Serverless Framework. Once user
settings are configured, it becomes operational. The platform supports all CRUD (Create, Read,
Update, Delete) operations for its main entities, such as customers, orders, and shipments. It
features a dashboard displaying key statistics and allows for the posting of operations on orders.
This triggers asynchronous processes like the creation of new shipments and invoicing for customers.
Technically, the chosen technologies lay a valid foundation for future enhancements and the
distribution of the platform across various cloud infrastructures. Despite encountering several
challenges during development, the thesis reaffirms the belief in the future of platforms like
Function as a Service, which streamline development by abstracting many underlying details. While
FaaS may not be suitable for all applications due to certain constraints, it offers significant
benefits for compatible systems, marking a progressive step in software development.
\newline\newline
Looking ahead, there are multiple areas for future development of the platform. Building on its
current capabilities, a wide array of ERP functionalities could be integrated, including warehouse
management, production, accounting, and human resources management. Additionally, the introduction
of a Command Line Interface (CLI) could offer a streamlined way to interact with the platform’s
code, enabling the automatic conversion of YAML files into templates for various cloud providers.
Another promising avenue for expansion is the incorporation of a NoSQL database like AWS DynamoDB or
Azure CosmosDB. This would allow for data storage in a truly serverless mode, enhancing the
platform’s serverless capabilities. Furthermore, the microservices architecture of the platform
lends itself well to the integration of a Machine Learning (ML) module. Such a module could analyze
platform data to provide valuable insights, such as identifying the best-selling products, top
customers, and leading suppliers.
\newline\newline
An additional area of future research could focus on the environmental implications of serverless
computing. In the context of growing concerns about energy consumption in data centers, serverless
computing, with its features like pay-per-use, auto-scaling, and multi-tenancy, could offer a more
sustainable solution. This aspect of serverless computing is particularly relevant in today’s world
and warrants further exploration to understand its potential in reducing the environmental footprint
of data centers.

% As already anticipated in the introduction, the objective of the thesis is to create a platform
% capable of handle the main operations of an ERP system and implementing design concepts and pattern
% of microservices and serverless computing, easily accessible via different platforms. At the end of
% the first version of the platform, we have few ERP specific functionalities but we have a technical
% base for continuing the development of different modules and functionalities. In fact, the platform
% has been designed to be easily extensible and scalable, so that it can be adapted to different needs
% and requirements. At the moment the platform can be deployed on an AWS tenant, thanks to Serverless
% Framework, and after setted the users, the platform is ready to be used. We can execute all CRUD
% operations on the main entities of the platform, such as customers, orders, shipments, etc., we have
% a dashboard that shows the main statistics of the platform, and we can execute a posted operation on
% an order, which will trigger the creation of a new shipment and the creation of an invoice for the
% customer, in an asynchronous way. Furthermore, from a technical point of view, we can confirm that
% the technologies used are valid foundations to then extend the functionalities with future
% developments and allow the platform to be distributed through different cloud infrastructures. even
% if I found several obstacles during the development of the platform, I remain convinced that the
% future for most developers is using a platform that hides much of the underlying detail from them,
% like FaaS. Not all applications will fit neatly into a FaaS ecosystem given the constraints, but for
% those that do, people are already seeing significant benefits. 

% With regard to future work, there are several fronts that can be developed. In fact, starting from
% the current state, a lot of ERP functionalities can be implemented, such as managing the warehouse,
% the production, the accounting, the human resources, etc. Furthermore, the platform can be extended
% with a CLI interface, which can be used to interact with the platform code and convert automatically
% the yml files into the corresponding cloud provider template. Another interesting development could
% be the implementation of a NoSQL database, such as AWS DynamoDB, or Azure CosmosDB, which can be
% used to store the data of the platform, but in a really serverless mode, making the platform totally
% serverless. Finally, due to the microservices nature, we could implement a ML module, which can be
% used to analyze the data of the platform and provide useful information to the user, such as the
% best selling products, the best customers, the best suppliers, etc. Another interesting future study
% could be the research about the green aspect of the serverless computing, which is a really
% important topic in the current world, and the serverless computing could be a really good solution
% to reduce the energy consumption of the data centers, thanks to the characteristics of the
% serverless computing, such as the pay-per-use model, the auto-scaling, and the multi-tenancy. 
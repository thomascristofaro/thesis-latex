\chapter{Kube Platform}
In this chapter, we commence on an in-depth exploration of the Kube platform, an implementation of
innovative ERP platform prototype. We start by outlining the fundamental requirements that have
shaped the development of the Kube platform, highlighting the strategic objectives, technical needs,
and business logic that inform its design. This leads us into a detailed examination of the
platform’s final architecture, which showcases a Microservices framework implemented through a
Function as a Service (FaaS) model, using AWS services. We then transition to practical
applications, where specific use cases demonstrate the platform's operational flow and its
real-world applicability. The chapter ends with a focus on the client application, delving into the
design and development of a user interface using Flutter, which not only complements the platform's
robust backend but also enhances the overall user experience. Throughout this chapter, we aim to
unravel the complexities of the Kube platform, illustrating its role as a transformative force in
the field of enterprise resource planning and highlighting its potential.

\section{Requirements}
Before starting development, requirements are established to define the properties of the product.
It's important that these requirements are thorough and coherent, covering all necessary features
without conflicts or inconsistencies. However, creating these documents can be challenging and
errors may occur, such as incomplete or ambiguous feature descriptions, redundancy, or important
details being omitted. To address these challenges, software engineering techniques have been
developed to formalize the requirements and minimize the occurrence of errors.
\newline\newline
ISO/IEC 25010\textsuperscript{\cite{ch5_1}}, an international standard, serves as a comprehensive
framework for assessing software product quality. This standard enumerates several key quality
characteristics, including functionality, reliability, usability, efficiency, maintainability,
security, compatibility, and portability. Its widespread application in software engineering and
quality assurance offers a standardized approach to evaluating and articulating software quality.
This facilitates more informed decisions in software acquisition, development, and maintenance.
ISO/IEC 25010 aids in identifying the actors involved and delineating both functional and
non-functional requirements.

\subsection{Stakeholders}
A stakeholder refers to any role, person, group, or organization that has an interest in a software
project or system being developed. This could include end-users, customers, investors, project
managers, developers and other individuals or groups involved in the development, deployment, and
maintenance of the software. Identifying all relevant stakeholders is important for considering
diverse perspectives and generating relevant requirements for the system. As shown in Table
\ref{tab:5_stakeholders}, numerous stakeholders play a role in the process.

\begin{table}[h]
    \centering
    \begin{tabular}{|l|p{10cm}|}
        \hline
        \textbf{Stakeholder} & \textbf{Description}                                                                                                                                                 \\ \hline
        End-users            & These are the people who will use the ERP system in their day-to-day work. They may include employees from various departments within the organization.              \\ \hline
        Developers           & These are the individuals responsible for creating the software code that makes up the ERP system.                                                                   \\ \hline
        Admin/IT staff       & These are the individuals responsible for installing, configuring, and maintaining the ERP system.                                                                   \\ \hline
        Customers            & These are the organizations or businesses that are purchasing the ERP system. They have a vested interest in ensuring the system meets their needs and requirements. \\ \hline
        Vendors              & These are the organizations that provide the ERP software and related services, such as installation, configuration, and support.                                    \\ \hline
        Cloud Vendors        & These hosts the system and provide the necessary infrastructure for its operation. They are responsible for system availability, scalability, and security.          \\ \hline
    \end{tabular}
    \caption{Stakeholders of a Cloud ERP System.}
    \label{tab:5_stakeholders}
\end{table}

\subsection{Functional and Non-functional}
Functional requirements and non-functional requirements are two types of requirements that are used
to specify what a system or software application should do and how it should perform. They are
important for the successful development and implementation of a system or software application. The
functional requirements ensure that the software application meets the needs of its users, while the
non-functional requirements ensure that the system is reliable, efficient, and secure.

\subsubsection{Functional requirements}
Functional requirements describe what the system should do in terms of its functionalities,
features, and capabilities. They define the specific tasks that the software application should be
able to perform to meet the needs of its users. For distinguish one requirement from another it is
important to assign for each functionality an ID, in order to easy identify it and trace throughout
the life cycle of the project (Table x).

% da inserire tabella
% Purchasing management: The system must be able to manage the entire purchasing process, from purchase requests to inventory management to actual purchase of products or services.
% Production management: The system must be able to manage the production process, including planning, scheduling, and control of production, as well as management of work orders.
% Logistics management: The system must be able to manage logistics, including storage, transportation, delivery, and tracking of goods.
% Sales management: The system must be able to manage the entire sales process, from quotes to inventory management to billing and post-sales support.
% Human resources management: The system must be able to manage HR processes, such as personnel management, training, performance, and human resources planning.
% Financial management: The system must be able to manage financial processes, including accounting, billing, budget management, bank activity management, and tax management.
% Project management: The system must be able to manage projects, from planning activities to resource management, to execution and control phases.

\subsubsection{Non-functional requirements}
Non-functional requirements describe how the system should perform in terms of its performance,
reliability, security, usability, and other aspects that are not directly related to the specific
functionalities to be implemented. They refer to operating methods and constraints, such as response
times, supported platforms, choice of languages, required resources, tools and various
implementation techniques. They must be measurable and may be more critical than functional
requirements. They are identified with a unique code and it is also necessary to specify their type
associated to the ISO properties and which functional requirements they refer to (Table x).

% quello che sta scritto qui non c'entra nulla vedi sotto
% \begin{table}
%     \centering
%     \begin{tabular}{p{2cm} | p{2.5cm} | p{4.5cm} | p{2.5cm}}
%         \hline\hline
%         ID         & Type & Description                                        & Refers to \\
%         \hline
%         Pages      & 3K   & 1 Page = 2 function request 1 to read - 1 to write & 6K        \\
%         Report     & 150  & 1 Report = 1 function req.                         & 150       \\
%         WebService & 76K  & 1 WS = 1 function req.                             & 76K       \\
%         \hline \hline
%     \end{tabular}
%     \caption{Example of ERP usage}\label{table_ERP_vendor}
% \end{table}
% da inserire tabella
% Scalability: The system should be able to handle a high volume of transactions and data, as it must be able to support all of the organization's functionalities.
% Reliability: The system should be highly reliable, as it must always be available for user access and use.
% Security: The system should be highly secure, as it manages sensitive business data, financial information, and personnel information.
% Usability: The system should be easy to use and intuitive for users, as ERP system users may have varying levels of computer proficiency.
% Performance: The system should be able to process and provide information in real-time, as business data must always be up-to-date.
% Maintainability: The system should be easily maintainable and modifiable, as business needs may change and require the addition of new functionalities.
% Interoperability: The system should be compatible with other software systems and technologies used by the organization, as the ERP system must be integrated with other business applications.


\section{Server application}
\subsection{Final Architecture}
% immagine architettura finale con l'integrazione dei servizi Amazon
\subsection{Database}
% un database SQL per ogni microservizio
% quali sono le tabelle

\section{Use Case}
\subsection{REST APIs}
\subsubsection{Security}
\subsubsection{User APIs}
\subsubsection{Customer APIs}
\subsubsection{Order APIs}
\subsubsection{Warehouse APIs}
\subsection{Async task}
\subsubsection{Posting order}
% Posting shipment
% Posting invoice

\section{Client Application}
\subsection{User Interface}

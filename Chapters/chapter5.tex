\chapter{Conclusions and future works}
% microservices_book
The way forward

I remain convinced that the future for most developers is using a platform that hides much of the
underlying detail from them. For many years, Heroku was the closest thing I could point to in terms
of something that found the right balance, but now we have FaaS and the wider ecosystem of turnkey
serverless offerings that chart a different path.

There are still issues to be ironed out with FaaS, but I feel that, while the current crop of
offerings still need to change to resolve the issues with them, this is the sort of platform that
most developers will end up using. Not all applications will fit neatly into a FaaS ecosystem given
the constraints, but for those that do, people are already seeing significant benefits. With more
and more work going into Kubernetes-backed FaaS offerings, people who are unable to make direct use
of the FaaS solutions provided by the main cloud providers will increasingly be able to take
advantage of this new way of working.

So, while FaS may not work for everything, it’s certainly something I urge people to explore. And
for my clients who are looking at moving to cloud-based Kubernetes solutions, I’ve been urging many
of them to explore FaaS first, as it may give them everything they need while hiding significant
complexity and offloading a lot of work.

I’m seeing more organizations making use of FaaS as part of a wider solution, picking FaaS for
specific use cases where it fits well. A good example would be the BBC, which makes use of Lambda
functions as part of its core technology stack that provides the BBC News website. The overall
system uses a mix of Lambda and EC2 instances—with the EC2 instances often being used in situations
in which Lambda function invocations would be too expensive.3
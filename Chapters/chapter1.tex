\chapter{Introduction}

\section{Overview of the Thesis}
Enterprise resource planning (ERP) refers to software used by
organizations to manage their day-to-day business activities such as
sales, procurement, accounting, manufacturing, and human
resources. These systems streamline organizational activities and
integrate data across departments, business units, and locations so
that each of these entities does not need its own business system.
The power of ERP systems is that they consolidate business data
into one centralized database that serves as a “single source of the
truth.” In practice, this means that employees in every area of the
enterprise can access the data needed to perform their jobs, and this
data is real-time and consistent across the company. Having one
“version of the truth” increases operational efficiency, improves
decision making, and reduces duplicate data. Organizations of every
size, in every industry, and all over the globe use ERP systems. ERP
knowledge is critical for business users, IT staff, or anyone that will
interact with them in any capacity. This chapter serves as an
introduction to ERP systems. It begins by defining ERP and outlining
key characteristics of ERP systems and the reasons for implementing
them, along with the drawbacks. The chapter concludes by charting
the evolution of ERP and describing the current ERP marketplace.
\chapter{Introduction}
In today's era of digital transformation, there's an urgent need for agile, efficient, and scalable
information systems in the corporate world, with Enterprise Resource Planning (ERP) systems playing
a critical role in integrating and optimizing business processes. However, the increasing complexity
and the requirement for flexibility highlight the limitations of traditional ERP architectures.
These platforms often rely on outdated technology, are built on monolithic architectures, and follow
archaic pricing models. Many such systems struggle with compatibility issues on modern operating
systems, suffer from obsolete graphical interfaces, and are not user-friendly.
\newline\newline
The aim of this thesis, therefore, is to develop an ERP platform called Kube that encapsulates ERP
functionalities within a modern technological framework. This includes seamless cloud integration
and an updated pricing model. The envisioned platform is designed to be cross-platform compatible,
ensuring accessibility across all current operating systems without the hassle of complex
installations. This approach is intended to address the shortcomings of traditional ERP systems by
offering a solution that is both technologically advanced and adaptable to the evolving digital
landscape.

\section{Overview of the Thesis}
% da rileggere
This document presents a thorough exploration of the key arguments and design choices essential to
achieving the set objectives. The thesis starts with the Chapter 2, an in-depth look into Enterprise
Resource Planning (ERP) systems. It starts with a discussion on the motivations behind implementing
ERP, highlighting the benefits and strategic advantages they offer. The chapter then critically
assesses the drawbacks of ERP systems, providing a well-rounded perspective on their limitations and
challenges. It also details the various modules of ERP systems and illustrates how they collaborate
to enhance business processes. A significant portion of this chapter is dedicated to the technology
underpinning ERP systems, with a special focus on the Three-Tier Client-Server Architecture.
Additionally, the market-related aspects of ERP systems are examined, including an analysis of the
costs involved in their implementation and maintenance, and a comparison of the competitive
landscape.
\newline\newline
Chapter 3 marks the beginning of our deep dive into the foundational narrative of our thesis. This
chapter introduces the key design concepts and architectural principles behind the Kube platform. It
emphasizes cloud computing as the cornerstone of the platform's operating environment. The chapter
also delves into the concept of microservices, an architectural approach that segments applications
into smaller, independently functioning services. A critical element of Kube's architecture is its
serverless framework, which promotes scalability and operational efficiency. The implementation of
sagas is discussed, outlining their role in managing failures and maintaining data consistency
across services. The chapter concludes with an exploration of event-driven architecture,
highlighting its importance in enabling reactive programming and facilitating responsive
interactions within the platform. These elements collectively define the architectural design of
Kube, with a focus on flexibility, scalability, and seamless service integration.
\newline\newline
Then, the Chapter 4 of the thesis transitions into the practical application, specifically focusing
on the technologies implemented in the Kube platform. Central to this are the AWS services, which
include Lambda for serverless computing, SQS and SNS for effective messaging, and RDS for efficient
database management. These services form a robust and scalable infrastructure essential to Kube's
functionality. The platform also capitalizes on serverless architecture to enhance resource
efficiency and minimize operational costs. Additionally, the Go programming language is employed for
its effectiveness in constructing high-performance applications. The user interface of Kube,
designed using Flutter, offers both versatility and aesthetic appeal, significantly improving the
platform's user experience. Collectively, these technologies are pivotal in ensuring Kube’s high
performance and user satisfaction.
\newline\newline
Finally, in Chapter 5, the focus shifts to the actual implementation of the application. The chapter
begins by delineating the fundamental requirements that have guided the Kube platform's development,
emphasizing strategic goals, technical necessities, and business rationale. This sets the stage for
a comprehensive exploration of the platform’s final architecture, which is marked by a Microservices
framework executed through a Function as a Service (FaaS) model, utilizing AWS services. The
narrative then moves to practical use cases, showcasing how the platform functions in real-world
scenarios. The chapter concludes by examining the client application, specifically the development
and design of its user interface using Flutter. This not only augments the robustness of Kube's
backend but also significantly enhances user interaction. This chapter aims to demystify the
complexities of the Kube platform, underlining its significance as a revolutionary tool in
enterprise resource planning and its vast potential in the field.
